\chapter{Introduction}
Data structures traversal can be achieved with several approaches and currently in Java it can be done in three different ways:
\begin{itemize}
\item Iterating elements with an Iterator object
\item Traversing elements individually (i.e. tryAdvance())
\item Traversing elements in bulk (i.e. forEachRemaining())
\end{itemize}
In java.util.stream the implementation of new query operations requires the definition of both individually and bulk traversal algorithms.
This incurs in undesired verbosity. In this project we propose a solution - Jayield - which infers the individual element traversal definition from a given bulk traversal method, and thus allows the implementation of new query operations with a single method (e.g. lambda expression).


\section{Motivation}

Java 8 brings with it the java.util.stream API (henceforth named just Stream) which allows us to perform queries fluently on a sequences of data, by providing methods that accept behaviour as parameters, allowing the programmers to express what their intentions are when calling the method.

\newpage
Let us consider the following two classes.
\lstinputlisting[caption={Fruit},label={lst:fruit}]{./Chapters/chapter1/listings/Fruit.java}
\lstinputlisting[caption={Basket},label={lst:basket}]{./Chapters/chapter1/listings/Basket.java}

Listing~\ref{lst:fruit} specifies the definition of a Fruit and listing~\ref{lst:basket} makes use of said definition and specifies the fruits inside a basket. Now, lets say we want to know the names of all the fruits inside the basket that have the color orange. In Java 7 and previous versions we would do something like this:

\lstinputlisting[caption={Collections approach},label={lst:basket-iterator}]{./Chapters/chapter1/listings/withIterator.java}

\newpage
As we can see in, we would attain an iterator and as we iterate over the fruits on the basket we check their color, if it matches our desired color we add it's name to the list and when we are done, we return the new list. Now lets look at how we can do this in Java 8 making use of Streams.

\lstinputlisting[caption={Stream approach},label={lst:basket-stream}]{./Chapters/chapter1/listings/withStream.java}

This approach lets us define our intentions, first we say we are only interested in fruits with the color orange, and that for each fruit that matches that description we are only interested in it's name. This approach lets us specify our intentions clearly which in turn makes the code easier to read and understand. 

However, the end use API is not the only difference between these two approaches. The collections approach is eager, meaning, it immediately goes through all the fruits in the basket to select the ones we are interested, taking note of the names as we go through them. The stream approach is lazy, as we said before, all we are doing is specifying our intention, no iteration is made on the sequence until it is consumed by a terminal operation, such as a forEach().

\section{Traversing sequences}

Going back to our initial example, the basket of fruits, we now have the names of all the orange fruits in our basket. Suppose that we want to take the names of the fruits in our basket alternately, skipping the even items and keeping the odd items. To that end we would like to have a filterOdd() operation chained in the pipeline of filter-map of listing ~\ref{lst:basket-stream}. However, the Stream API that Java 8 provides does not support this operation, so we have to implement it ourselves.

Java 8 Provides different ways of creating a new Stream. Since we want to preserve the laziness characteristic of streams we should implement the new Stream through a traversal protocol. In this case, Java 8 introduces a new iteration approach specified by the Spliterator interface which we can implement, as well as providing the AbstractSpliterator class that we can extend.

In listing~\ref{lst:spliterator} we show an example of the implementation of a filterOdd operation (line~\ref{line:filterOdd}) which in turn requires an auxiliary class implementing spliterator (i.e. OddFilter, lines~\ref{line:newoddFilter} and~\ref{line:oddFilter}). In this case we extend the AbstractSpliterator instead of implementing the Spliterator interface to reduce the number of methods we're required to implement.

\lstinputlisting[caption={Spliterator example},label={lst:spliterator}]{./Chapters/chapter1/listings/Example.java}


This approach is verbose, and only benefits from bulk traversing because we have overriden forEachRemaing(), as the Spliterator documentation states, the forEachRemaining()~\citep{spliterator} method's default implementation should be overridden whenever possible as it just calls tryAdvance() until it returns false, in other words, it will traverse element by element instead of the whole sequence. We will be discussing this approach further in Chapter 2. 

To overcome the shortcomings found in the Stream API we will implement a solution that allows the user to define a single way to traverse a sequence and we will generate it's counterpart. In this case the filterOdd() operation can be expressed in a lambda expression chained with the sequence of items as shown in listing~\ref{lst:jayield-filterOdd}.

\lstinputlisting[caption={Jayield filterOdd},label={lst:jayield-filterOdd}]{./Chapters/chapter1/listings/JayieldFilterOdd.java}

The programmer just needs to write one method to be able to traverse a sequence as the individually advance will be inferred from the previous definition.


\section{Use Case}
The use case that will be analysed, developed and tested will be the generation of an Advancer from either a Query or a Traverser.

The difficulty with this use case is the fact that a Traverser is used to bulk traverse an entire sequence, meaning it is not (or should not be) prepared to verify if it can advance at each element of the sequence. To do this we will copy the Traverser's code and instrument the required parts to generate a new Advancer. To test if the generated Advancer is working properly we will verify that each call to tryAdvance that succeeds, yields one element and only one element.
